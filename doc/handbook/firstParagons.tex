In this section you find how to install the Paragon compiler \texttt{parac}
and some first programs that handle information flows in simple code. In later
chapters we have a look at more complicated flows involving methods, exceptions
and Java classes.

\section{Installation}

Paragon is an operating system independent compiler. To install paragon you need
to have the following programs already installed:

\begin{itemize}

  \item The Haskell Platform
  
  \item Cabal
  
  \item Happy and Alex

\end{itemize}

Paragon can then easily be installed using cabal:

\begin{lstlisting}
cabal update
cabal install paragon
\end{lstlisting}

Congratulations, you just installed the paragon compiler!

\section{Explicit information flows}

We start with a small program that only knows two policies. Policy 
\texttt{top}, specifying that information can flow to no one, and policy
\texttt{bottom}, specifying that information can flow to everyone. As described
in Chapter~\ref{chap:paralocks}, these two policies are specified in Paralocks as
follows:

\begin{lstlisting}
top    = { }
bottom = { 'x : }
\end{lstlisting}

In order to specify these policies within Java, Paragon extends Java with a new
primitive type \newSyntax{policy}. At declaration, 
a variable can be given a policy by writing a 
question-mark followed by the name of the policy next to the type of 
the variable. This policy is called the \emph{read effect} of the variable.
That is, it specifies what actors may know the value of this variable. During
compilation Paragon checks that the value of this variable never affects 
variables with a less restrictive policy.

To run a first test of Paragon, create a file called 
\smallcode{MyVeryFirstParagon.para} and copy the following code (or use the file
provided in the \smallcode{codesrc} directory that should come with this handbook).

\begin{paragoncode}
public class MyVeryFirstParagon {

  public static final policy top = { : } ;
  
  public static final policy bottom = { 'x : } ;
  
  public void assignUp() {
  
    ?top int highData = 0;
    ?bottom int lowData = 0;
    
    highData = lowData;
    
  }
  
  public void assignDown() {
  
    ?top int highData = 0;
    ?bottom int lowData = 0;
    
    lowData = highData;
    
  }

}
\end{paragoncode}

In this example, we create two simple methods. Each of them has a variable with
high, secret data and a variable containing low, public data. In the method
\smallcode{assignUp} the public data is copied into the variable which carries secret
data. This is an allowed flow, since no one gains any more knowledge. In the
method \smallcode{assignDown} however, the secret data flows to the public and the
data that was first known to nobody, now becomes known to everyone. This 
\emph{explicit} flow of information should be detected by Paragon\footnote{
In practise, there is no need to annotate local variables (i.e. within methods)
in Paragon. These policies will be automatically derived by the compiler. However,
for the sake of creating simple example we use this here.
}.

In a terminal, navigate to the directory where \smallcode{MyVeryFirstParagon.para}
is located. To compile the file, run (assuming the cabal \smallcode{bin} directory
is in your \smallcode{PATH} environment variable):

\begin{lstlisting}
parac MyVeryFirstParagon.para
\end{lstlisting}

Depending on the synchronization between this handbook and the compiler, you 
might get the error:

\begin{lstlisting}
Unknown package: java.lang
\end{lstlisting}

In order to combine several classes and preserve the information flows between
instances of those classes, Paragon requires each imported class to have an
interface file. This interface file contains information on the policies and
information flows in the imported class and is used by Paragon to check if no
information is leaked by using instances of this class. We discuss this concept
in more detail in Chapter~\ref{chap:interfaces}. 

The reason you get this error is that Paragon, like Java, automatically assumes
the presences of all classes in the \smallcode{java.lang} package. Hence, to compile 
any paragon program, the interface files for this package have to be in the
\emph{PI-path} of \smallcode{parac}. A bunch of interface files has been installed
with Paragon, and you can simply add the directory \emph{TODO} to this path.

The PI-path can be modified both by setting the environment variable \smallcode{PIPATH}
as well as by parsing it as an argument to the compiler flag \smallcode{-p}. For the
remainder of this chapter we assume that the correct path resides in the 
environment variable.

When successful, the compiler outputs:

\begin{lstlisting}
When checking class MyVeryFirstParagon:
When checking body of method assignDown:
In the context of lock state: []
Cannot assign result of expression highData with policy {:} to location lowData with policy {'x:}
\end{lstlisting}

Currently, the compiler only outputs to the console when an error occurred. In
this case \smallcode{parac} correctly detected the invalid information flow in the 
method \smallcode{assignDown}. Comment out line \smallcode{21} in \smallcode{MyVeryFirstParagon.para}
to remove this information flow and compile the file again.
Compilation is now successful and the files \smallcode{MyVeryFirstParagon.pi} and
\smallcode{MyVeryFirstParagon.java} are generated. The \smallcode{.java} file can be compiled
to be run together with the Paragon runtime library (more on this in 
Chapter~\ref{chap:runtime}), the \smallcode{.pi} file is used when compiling different
classes that use this class. If you are interested you can have a look at these
files, but for the moment they are not relevant.

When combining data from several sources, the policy of each source needs to be
taken into consideration. For example, in the assignment:
\begin{lstlisting}
lowData = lowData + highData;
\end{lstlisting}
information flows from both \smallcode{top} and \smallcode{bottom} policy-annotated data. 
The policy on the expression needs to be what is called the 
\emph{least upper bound} of these two policies. Intuitively, this is the policy
that allows information flows only if they are allowed by each of the individual
policies. Therefore the combination of data can only give rise to more restrictive
policies. For an exact description of the least upper bound operator (including 
actors and locks), see Appendix~\ref{app:paralocksOperations}.

In this example, the policy of the expression \smallcode{lowData + highData} becomes
\smallcode{top}. If you add this expression to your Paragon file and compile it, you
get the error:

\begin{lstlisting}
When checking class MyVeryFirstParagon:
When checking body of method combineData:
In the context of lock state: []
Cannot assign result of expression highData + lowData with policy {:} to location lowData with policy {'x:}
\end{lstlisting}

Here it is possible to give an example of the incompleteness of Paragon. Paragon
does not consider the semantic structure of expressions, hence in the following
assignment:
\begin{lstlisting}
lowData = highData - highData;
\end{lstlisting}
the policy derived by Paragon on the right hand side is still \smallcode{top}, despite
the result of the computation always being 0.

\section{Implicit information flows}

As described in Chapter~\ref{chap:introduction} information flows can present
themselves in multiple ways. Where the previous section demonstrated flows 
directly from one variable into the other, here we consider indirect flows of
information. The following example is famous for demonstrating implicit flow
in its most simple form:

\begin{paragoncode}
public class ImplicitFlows {

  public static final policy top = { : } ;
  
  public static final policy bottom = { 'x : } ;

  public void simpleFlow() {
  
    ?top boolean highData = true;
    ?bottom boolean lowData = false;
    
    if (highData) {
      lowData = true;
    } else {
      lowData = false;
    }
    
  }
}
\end{paragoncode}

Clearly we are copying the value of \smallcode{highData} into
\smallcode{lowData}, but we do not use a direct assingment for that. Paragon
uses the concept of a \emph{program counter}: a policy that represents the
lower bound on what assignments can be executed. In this example the program
counter is increased to the policy of \smallcode{highData}, \smallcode{top},
for each of the two branches of the condition. As a result, only variables with
a policy of \smallcode{top} or higher may be affected in these branches. This 
concept is revisted in Chapter~\ref{chap:methods} on method signatures.

Paragon gives you the following error on compiling this code:

\begin{lstlisting}
When checking class ImplicitFlows:
When checking body of method simpleFlow:
Assignment to lowData with policy {'x:} not allowed in branch dependent on condition highData with write effect bound {:}
\end{lstlisting}

After typechecking the conditional statement, the program counter is reset 
to its value before the conditional statement, so
that assignments to variables with a policy lower than \smallcode{top} is 
possible after the conditional statement.

A conditional assignment on the other hand, such as below, is considered as a
direct flow. That is, the occurrence of the variable \smallcode{highData} on
the right hand side of the assignment makes that its policy is part of the least upper
bound computed for the whole expression. 

\begin{paragoncode}
lowData = highData ? true : false;
\end{paragoncode}

Although we are specifying policies on local variables here, this is not needed
in general. Paragon will try to infer the policies based on context information
and give an error message if this is not possible (which in general implies that
there is an illegal flow of information). Currently these errors are still a bit
cryptic. Consider the following, more complicated information flow:
\begin{paragoncode}
  public void moreComplicated() {
  
    ?top boolean highData = true;
    ?bottom boolean lowData = false;
    
    boolean temp = false;
    if (highData)
      temp = true;
            
    if (temp)
      lowData = true;
      
  }
\end{paragoncode}

When observed carefully, you see that the value of \smallcode{highData} always
gets copied to \smallcode{lowData}, via two implicit flows using the additional
variable \smallcode{temp}. Here we left the policy on the variable \smallcode{temp}
unspecified and leave it to Paragon to derive it. Compilation gives us:

\begin{lstlisting}
When checking class ImplicitFlows:
When checking body of method moreComplicated:
The system failed to infer the set of unspecified policies
\end{lstlisting}

Paragon tells us it is unable to derive the unspecified policies, in this case
that has to be the one on \smallcode{temp}. Indeed, we cannot associate the
policy \smallcode{bottom} with \smallcode{temp}, since then we have an implicit
flow of information in line \smallcode{7-8}.  Neither can we associate the
policy \smallcode{top}, since then we have an implicit flow in line
\smallcode{10-11}.

Up to this point we have been considering programs where the impicit flow leaks
the entire value of the data to be protected. In general implicit flows leak
less information. The following example only leaks the parity of 
\smallcode{highData}:

\begin{paragoncode}
  public void leakParity() {
  
    ?top int highData = 13;
    ?bottom int lowData = 0;
    
    if (highData % 2 == 1)
      lowData = 1;
      
  }
\end{paragoncode}

Implicit flows also result from while loops and exceptions, the latter we 
consider in Chapter~\ref{chap:methods}.

\section{Introducing actors}

\section{Unlocking a world of Locks}
